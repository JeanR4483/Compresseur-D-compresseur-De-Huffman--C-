\begin{tad}
 \tadNom{ArbreDeHuffman}
 \tadParametres{Donnee (possède un ordre total)}
 \tadDependances{\booleen, \naturelNonNul}
 \begin{tadOperations}{obtenirDonnee}
 \tadOperation{feuille}{Donnee}{\tadParams{ArbreDeHuffman}}
\tadOperation{estUneFeuille}{\tadParams{ArbreDeHuffman}}{\tadParams{Booleen}}
 \tadOperationAvecPreconditions{obtenirFilsGauche}{\tadParams{ArbreDeHuffman}}{\tadParams{ArbreDeHuffman}}
\tadOperationAvecPreconditions{obtenirFilsDroit}{\tadParams{ArbreDeHuffman}}{\tadParams{ArbreDeHuffman}}
\tadOperationAvecPreconditions{obtenirDonnee}{\tadParams{ArbreDeHuffman}}{\tadParams{Donnee}}
\tadOperation{obtenirPonderation}{\tadParams{ArbreDeHuffman}}{\tadParams{\naturelNonNul}}
\tadOperation{ajouterRacine}{\tadParams{ArbreDehuffman, ArbreDeHuffman}}{\tadParams{ArbreDeHuffman}}
\end{tadOperations}
\begin{tadSemantiques}{longueur}
\tadSemantique{feuille}{crée un arbre de Huffman (feuille) à partir d'une donnée}
\tadSemantique{ajouterRacine}{rassemble deux arbres et calcule la pondération associée (somme des pondérations/données des deux sous-arbres)}
\end{tadSemantiques}
\begin{tadAxiomes}
\tadAxiome{estUneFeuille(feuille())}
\tadAxiome{non(estUneFeuille(ajouterRacine(ag,ad)))}
\tadAxiome{obtenirDonnee(feuille(d))=d}
\tadAxiome{obtenirFilsGauche(ajouterRacine(ag,ad)) = ag}
\tadAxiome{obtenirFilsDroit(ajouterRacine(ag,ad)) = ad}
\end{tadAxiomes}
\begin{tadPreconditions}{obtenirDonnee(a)}
\tadPrecondition{obtenirFilsGauche(a)}{non estUneFeuille(a)}
\tadPrecondition{obtenirFilsDroit(a)}{non estUneFeuille(a)}
\tadPrecondition{obtenirDonnee(a)}{estUneFeuille(a)}
\end{tadPreconditions}
\end{tad}

    